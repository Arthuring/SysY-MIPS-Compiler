\chapter{目标代码生成}
\section{设计概述}

此阶段将体系结构无关的中间代码翻译成MIPS体系结构的目标代码。程序的控制流,计算指令等已经在中间代码生成阶段完成,
对于单条中间代码如何翻译成目标代码难度不大。目标代码生成的难点主要集中在寄存器分配和管理,运行时的存储管理,切换和恢复运行现场。

本编译器生成目标代码的流程如图\ref{fig:translator}。

\begin{figure}[htbp]
	\centering
	\includegraphics[width=0.8\linewidth]{img/translator.png}
	\caption{生成目标代码基本流程}
	\label{fig:translator}
\end{figure}

大致步骤如下:
\begin{enumerate}
    \item 首先将要输出的字符串填在data段,设置全局变量初值并设置全局指针\texttt{\$gp}和帧指针\texttt{\$fp}。
    \item 随后生成函数体代码,在函数入口处需要计算并分配函数的局部变量地址,同时压栈设置栈指针\texttt{\$sp}。
    \item 若在优化中采用了寄存器传参和图着色寄存器分配,需要取参数。
    \item 随后对每个基本块代码生成,基本块入口处需要计算并分配临时变量空间,压栈指针,离开基本块时回收这部分空间。
    \item 生成函数出口,将函数刚刚分配的局部变量空间回收,若是main函数,则结束程序,否则跳转到返回地址。
\end{enumerate}

\section{设计细节}

\subsection{寄存器分配}

采用一个单独的类\texttt{RegMap}来维护寄存器的分配。\texttt{RegMap}定义如下
\begin{minted}{java}
public class RegMap {
    private static final Map<Integer, TableEntry> BUSY_REG_TO_VAR = new HashMap<>();//寄存器到变量的映射
    private static final Map<TableEntry, Integer> VAR_TO_BUST_REG = new HashMap<>();//变量到寄存器的映射

    private static final Collection<Integer> availableReg = Collections.unmodifiableList(Arrays.asList(
            8, 9, 10, 11, 12, 13, 14, 15, 16, 17, 18, 19, 20, 21, 22, 23, 24, 25
    ));
    private static final Set<Integer> freeRegList = new HashSet<>(availableReg);//目前空闲的寄存器
    private static final Set<Integer> lruList = new LinkedHashSet<>();//LRU队列
    private static final Map<Integer, Boolean> REG_DIRTY = new HashMap<>();
    //Dirty位,寄存器中的值是否被修改过

     /**
     * 分配寄存器,若已经分配过,则返回之前分配的,并更新LRU。
     * 若未分配,则分配一个,若需要分配同时加载初值,则needLoad置位。
     * 副作用:会在mipsObject中新增代码
     */
     public static int allocReg(TableEntry tableEntry, MipsObject mipsObject, boolean needLoad) {
     }
     //...
     }
     //...
\end{minted}

若没有空闲的寄存器,采用最近最少使用的寄存器置换方法,将最近最少使用的寄存器释放,分配给待分配变量。
若该寄存器的值的\texttt{Dirty}为\texttt{true},则需要写回对应内存。

\subsection{运行时存储管理}
运行栈的结构设计如图\ref{fig:stack}。
\begin{figure}[htbp]
	\centering
	\includegraphics[width=0.5\linewidth]{img/stack.png}
	\caption{运行栈结构设计}
	\label{fig:stack}
\end{figure}


通过\texttt{\$gp}访问全局变量,通过\texttt{\$fp}访问局部变量和参数,通过\texttt{\$sp}访问临时变量。

\subsection{函数调用的现场切换与恢复}

发生函数调用时,需要在运行栈上保存好函数的返回地址、分配给局部变量的全局寄存器和当前函数的帧指针\texttt{\$fp},并在函数调用后恢复。
由于函数调用并不会引起基本块的分割,有些临时变量需要在函数调用后维持原本的值,因此在发生函数调用时,还需要扫描基本块的后续指令,判断哪些
临时变量还需要用到,将这些临时变量保存在内存对应的位置。其余的不被用到的临时变量可以直接清除映射。

\subsection{数组寻址的计算}

从中间代码的\texttt{getelementptr}指令计算访问地址也是一个难点。本编译器中的\texttt{ElementPtr}定义如下
\begin{minted}{java}
public class ElementPtr extends InstructionLinkNode {
    private final TableEntry dst; //计算出的地址的保存变量
    private final TableEntry baseVar;//基变量
    private final List<Operand> index = new ArrayList<>();
    //地址运算的index

    //...    
}
\end{minted}

地址的计算公式为

$Addr_{baseVar} + 4\Sigma_{i = 0} ^ {index.size} (i < baseVar.dim.size ? baseVar.dim[i] : 1) * index[i]$。

其中对于\texttt{baseVar}为数组的访问要比\texttt{baseVar}是指针的访问index在最前多出一个0。
例如
\begin{minted}{java}
    int a[4][5], b[5];

    a[2][3] = 1;
    //为访问a[2][3]产生的index序列为 0, 2, 3
    //计算出的offset为4*(4*0+5*2+1*3)=4*13
    b[3] = 1;
    //为访问b[3]产生的index序列为 0, 3
    //计算出的offset为4*(5*0+1*3)=4*3
    void foo(int a[][5],int b[]){
        a[2][3] = 1;
        //为访问a[2][3]产生的index序列为 2, 3
        //参数的第一维缺失,dimension中只有5
        //计算出的offset为4*(5*2+1*3)=4*13
        b[3] = 1;
        //为访问b[3]产生的index序列为 3
        //参数的第一维缺失,dimension中为空
        //计算出的offset为4*(1*3)=4*3
    }
\end{minted}

这个计算方法对于更高维的数组也是适用的,若要拓展数组维数,此架构并不需要修改。

\section{编码后的修改}
本阶段的主要修改是添加了代码优化后带来的修改,由于加入了图着色寄存器分配和寄存器传参,因此多出了取参数和
注册图着色分配的寄存器这一步骤,其他部分大体上没有改动。