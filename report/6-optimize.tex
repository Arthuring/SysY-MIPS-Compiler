\chapter{代码优化}
\section{体系结构无关优化}

\subsection{流图建立}
流图建立是所有优化的基础,在本编译器中,中间代码生成部分就已经完成了基本块的划分,
因此在此只需要把他们的关系建立起来即可。

流程如下
\begin{enumerate}
    \item 首先在函数体中维护一个从\texttt{basicBlockLabel}到\texttt{basicBlock}的\texttt{Map}。
    \item 在函数体最后新增一个空的出口基本块。
    \item 在每个基本块内部维护前驱基本块标签集和后继基本块标签集两个\texttt{Set}。
    \item 若基本块最后一条中间代码既不是分支也不是跳转也不是返回,则将此基本块直接相连的下一个基本块加入后继集。同步维护下一个基本块的前驱集。
    \item 若基本块的最后一条是跳转,则将跳转目标Label加入后继集,同步维护目标基本块的前驱集。
    \item 若基本块最后一条是分支,则将真假两个目标Label加入后继集,并同步维护目标基本块的前驱集。
    \item 若基本块最后一条指令是返回,则将出口块加入后继集。
    \item 若基本块是最后一个基本块,则将出口加入后继集。
\end{enumerate}

\subsection{合并基本块}

\subsection{窥孔优化}

\subsection{到达定义分析}

\subsection{常量传播}

\subsection{复写传播}

\subsection{基本块内部的死代码删除}

\subsection{活跃变量分析}

\subsection{跨基本块的死代码删除}

\subsection{循环结构优化}

\section{体系结构相关优化}

\subsection{图着色寄存器分配}

\subsection{乘除优化}

\subsection{指令选择优化}