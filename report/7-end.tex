\chapter{结语}
一学期的编译实验终于进入尾声,回想整个过程,比较深刻的感受可以用“细节繁杂”一词来概括。本学期的编译实验,尤其是调试的时候,
总会给心态带来冲击。

由此,在开始编写文法复杂,细节繁多的编译器的各个模块前,一个整体的设计是十分必要的,先设计再编写是避免一开始就落入繁杂细节的好方法,
大方向上的正确可以有效避免编码过程中的大规模重构,出现的细节问题也只是多,杂,但往往并不致命,只需要稍作修改就可以解决。当然,
完美无瑕的整体设计也很难做到,随着编写不断推进,理解不断深入,一些新的问题在这时才会暴露出来,因此在最初设计时,要尽量减少不同模块之间的
耦合度,让修改更加容易。

在优化方面,印象比较深刻的感受是付出有时并不能带来回报,或者说只有不断积累量变才可能带来质变。由于优化的效果和测试数据的结构极为相关,
在一次次实现新优化后的一次次提交中,有时会发现这项新优化效果并没有反应在测试数据上,不免会感觉十分失落。有时完成一项很简单的优化后,反而
效果出乎意料很好,这有可能是这项优化自身产生的效果,也可能是和之前优化共同作用的结果。因此需要调整面对竞速点的心态,竞速点涉及到的测试点很少,
并且是刻意构造过的,即使某项优化在这几个竞速点中没有体现作用,在其他的情况中也许就能有所发挥,成功调通,实现一项优化本身就是值得高兴的事了。

当然,建议课程组可以增加一些竞速点,刻意构造过的针对某项优化的测试点很难真实反映优化效果,确实会在过程中带来一些挫败感。