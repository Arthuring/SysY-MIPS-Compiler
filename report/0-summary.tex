\chapter{总体架构}
本编译器的总体架构如\ref{fig:summary}。

\begin{figure}[htbp]
	\centering
	\includegraphics[width=0.8\linewidth]{img/summary.png}
	\caption{编译器总体架构}
	\label{fig:summary}
\end{figure}

本编译器由以下几个主要模块组成
\begin{itemize}
    \item 词法分析器:读入源代码,进行词法分析,输出一个token流。
    \item 语法分析器:利用语法分析器递归下降分析token流,得到具体语法树,遇到错误时添加到全局错误表。
    \item 语义分析器:利用语义分析器进行符号表的管理,并得到抽象语法树,遇到错误时添加到全局错误表。
    \item 中间代码生成器及优化器:得到抽象语法树后根据不同的语法节点类型生成llvm中间代码,并且可以配置是否开启优化。
    \item 目标代码生成器及优化器:通过llvm中间代码生成mips目标代码,可以配置选择是否开启优化。
\end{itemize}

随后的章节将会介绍这些模块的设计细节。


