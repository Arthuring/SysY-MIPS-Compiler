\chapter{语法分析}
语法分析的任务是遍历词法分析得到的单词有序表,根据给定的形式文法,分析并确定其语法结构。

本编译器采用了语法树这种层次化的结构保存语法分析的结果,
单词有序表经过语法分析后,将得到一个完全符合课程给出文法的具体语法树。
为了避免过于冗余的代码以及满足语法分析的输出要求,本编译器在本阶段的语法分析中,
得到的所有语法成分(终结符以及非终结符)均采用统一的\texttt{CompileUnit}类来表示,
通过\texttt{CompileUnit}类中的\texttt{name}和\texttt{type}来区分不同的成分,
用\texttt{isEnd}成员来标记是否为终结符。类的定义如下。
\begin{minted}{java}
public class CompileUnit {
    private final String name;//若为非终结符,则为类型名,否则为终结符内容
    private final Type type;//语法成分类型
    private final List<CompileUnit> childUnits;//语法子树
    private final boolean isEnd;//是否是终结符
    private final Integer lineNo;//若为终结符,则需要有行号
    ...
}
\end{minted}   