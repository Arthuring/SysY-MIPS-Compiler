\chapter{语法分析}
\section{设计概述}
语法分析的任务是遍历词法分析得到的单词有序表,根据给定的形式文法,分析并确定其语法结构。

本编译器采用了递归下降的方法进行语法分析,采用语法树这种层次化的结构保存语法分析的结果,
单词有序表经过语法分析后,将得到一个完全符合课程给出文法的具体语法树。
为了避免过于冗余的代码以及满足语法分析的输出要求,本编译器在本阶段的语法分析中,
得到的所有语法成分(终结符以及非终结符)均采用统一的\texttt{CompileUnit}类来表示,
通过\texttt{CompileUnit}类中的\texttt{name}和\texttt{type}来区分不同的成分,
用\texttt{isEnd}成员来标记是否为终结符。类的定义如下。
\begin{minted}{java}
public class CompileUnit {
    private final String name;//若为非终结符,则为类型名,否则为终结符内容
    private final Type type;//语法成分类型
    private final List<CompileUnit> childUnits;//语法子树
    private final boolean isEnd;//是否是终结符
    private final Integer lineNo;//若为终结符,则需要有行号
    ...
}
\end{minted}

\section{设计细节}

\subsection{文法左递归处理}
文法左递归会给自顶向下的递归下降语法分析方法带来无限递归或不可避免的回溯问题,因此需要对文法进行BNF范式改写。
文法中的左递归主要出现在表达式部分,以\texttt{AddExp}为例,\newline
\texttt{AddExp →  MulExp |AddExp ('+' | '−') MulExp } 可以改写为

\texttt{AddExp →  MulExp { ('+' | '−') MulExp}}。其他表达式相关文法均按照类似方法消除左递归。

需要注意的是,语法分析作业需要检查语法成分的输出顺序。由于改写文法的同时也造成语法树的改动,即由二叉树转化成了多叉树,因此这种改动可能会造成输出顺序与要求不同,
所以还需要将识别到的语法成分转化回原来的语法树结构。仍以\texttt{AddExp}为例,具体方法是每当读到一个\texttt{+/-}时,就将读到加减号前的语法结构向上打包成一个新的\texttt{AddExp},
这样在输出时就不会由于多叉树公用根的问题导致缺少输出了。

\subsection{赋值语句与表达式语句区分问题}

文法中的\texttt{Stmt → LVal '=' Exp ';' }和\texttt{Stmt → [Exp] ';'}的FIRST都有可能是\texttt{LVal},因此只向前看一个字符很难确定要用使用什么规则递归下降,
为了较好的和原语法树结构相符,此处本编译器没有做特殊处理,而是采用向前看两个符号的方法,
先尝试读入一个\texttt{LVal},若能读入一个\texttt{LVal},则看下一个词法成分是否是\texttt{=},如果是,则为赋值语句。

\subsection{为错误处理预留接口}

在进行语法分析设计时,为错误处理预留了接口,当解析到不符合文法规则的成分时,会抛出异常。

\section{编码后的修改}

\begin{enumerate}
    \item 在初次实现时,判断应用哪条规则时采用了当前符号“不是...”判断,比如当前不是分号,就继续读入\texttt{LVal}等,
    这样的设计在错误处理中带来了问题,因为被用来判断的符号可能缺少,例如缺分号错误,导致程序出错。因此在最后
    判断条件都改为了当前词法元素“是...”来判断。
\end{enumerate}