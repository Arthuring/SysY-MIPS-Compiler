\chapter{中间代码生成}
\section{设计概述}

中间代码生成的任务是将树状结构的抽象语法树,转化成线性结构的中间代码序列。本编译器的中间代码采用了llvm中间代码。

本编译器将中间代码分为了12类,如表\ref{table:midcode}。

生成中间代码的过程就是遍历抽象语法树,并将语法树中嵌套的\texttt{block}转化成中间代码中线性的\texttt{basicBlock}。

划分基本块有助于将嵌套结构转化成线性,因此本编译器在此阶段完成基本块划分,也为之后代码优化打基础。

保存中间代码的数据结构方面,采用了链表来保存中间代码序列,
便于后续在代码优化阶段会频繁的发生代码的增删,替换操作。

\begin{table}[H] 
    \centering
    \begin{tabular}{ccc}
    \toprule 
    类型 & 意义 & 样例 \\
    \midrule
    funcDef & 函数定义 & \texttt{define i32 @fib(i32 \%i\_1)} \\
    VarDef & 变量定义  & \texttt{\%i\_1 = alloca i32} \\
    BinaryOperator &  二元运算 & \texttt{\%-t15\_0 = mul i32 \%-t13\_0, \%-t14\_0}\\
    UnaryOperator &  一元运算 & 转化为二元形式输出 \\
    Branch & 分支 & \texttt{ br \%-t106\_0 label \%label\_11 label \%label\_9}\\
    Jump & 跳转 & \texttt{br label \%while\_cond\_label\_10 }\\
    Call & 函数调用 &\texttt{\%-t7\_0 = call i32 @fib(i32 \%-t9\_0) }\\
    ElementPtr & 数组寻址  &\texttt{getelementptr [10 x i32], [10 x i32]* @a, i32 0, i32 1 }\\
    PointerOp  &  内存操作(存取)&\texttt{\%-t35\_0 = load i32, i32* \%k\_1 }\\
    Return & 返回 &\texttt{ret i32 0 } \\
    PrintInt & 输出整数 & \texttt{call void @putint(i32 \%-t109\_0 ) }\\
    PrintStr & 输出字符串 & \texttt{ call void @putch(i32 44 )    ; ','}\\
    \bottomrule %添加表格底部粗线
    \end{tabular}
    \caption{中间代码表}
    \label{table:midcode}
\end{table}

\section{设计细节}

\subsection{变量的定义和初始化}

对于全局变量,加入中间代码的全局变量列表,并将符号表表项中的\texttt{isDefined}设置为\texttt{true}。

对于局部变量,在代码序列中加入一个\texttt{VarDef},并将符号表表项中的\texttt{isDefined}设置为\texttt{true}。

对于数组,由于数组的维数定义目前还是常量表达式,需要对数组维度定义进行常量表达式化简。

对于初始值,普通变量的初始值只有一个,在进行常量表达式化简后,在代码序列中加入一个\texttt{STORE}即可。数组变量的初值
由于存在嵌套,需要用广度优先遍历得到线性的初值序列,然后通过\texttt{STORE}依次赋值。

对于常量,非数组类的常量都可以在此阶段直接替换为数,不用再以变量的形式出现,因此中间代码中不会出现对于非数组类常量的定义。
数组类常量则是可以在此阶段直接确定通过下标访问值,但是对于通过变量访问的值则无法确定,因此仍然需要出现在中间代码中。

样例:
\begin{minted}{java}
%i_1 = alloca i32
store i32 2, i32* %i_1
\end{minted}

\subsection{变量的访问和赋值}
变量表的表项掌握着变量的所有相关信息,因此可以直接将变量表的表项当作变量使用。
由于中间代码阶段还没有寄存器分配,目前所有的变量都分配在内存上,因此对变量的访问和赋值都需要通过\texttt{PointerOp(STORE,LOAD)}进行。
不同层定义的同名变量本质上是不同的变量,因此应该加以区分,同时也方便之后的代码优化,
因此需要重写变量表表项的\texttt{Equal}方法,将名字和层数都相等的变量视为同一个变量。

\begin{minted}{java}
    public boolean equals(Object o) {
        if (this == o) return true;
        if (o == null || getClass() != o.getClass()) return false;
        TableEntry that = (TableEntry) o;
        return level == that.level && Objects.equals(name, that.name);
    }
\end{minted}

变量的访问需要先通过符号表查找被标记为\texttt{isDefined}的变量,
然后新增一个临时变量来保存变量的值,在代码序列中加入一个\texttt{PointerOp(LOAD)},返回这个临时变量。

对变量的赋值可以是赋值一个立即数,也可以赋值一个临时变量中保存的结果,先通过符号表查找被标记为\texttt{isDefined}的变量,
然后在代码序列中加入一个\texttt{PointerOp(STORE)}。

\subsection{表达式计算}

首先表达式可以先经过一次常量化简,将常量都替换成立即数。
表达式计算的基本逻辑是将已有的变量或立即数作为操作数,然后新增一个临时变量来存放表达式的计算结果,并返回这个存结果的临时变量。
调用时,递归地调用即可。

\todo 加入一个表达式计算的图。

\subsection{控制流(循环和分支)}
\subsubsection{分支}
对于分支,也就是\texttt{if else}语句:

\begin{minted}{java}
    if(cond){
        //ifStmt
    }else{
        //elseStmt
    }
    //end, new basicBlock
\end{minted}
按照如下方式生成

\begin{minted}{java}
    temp = cond;//temp保存表达式cond的计算结果
    br temp label ifStmt_label label elseStmt_label
    elseStmt_label:
        //elseStmt
        br label newBasicBlock_label
    ifStmt_label:
        //ifStmt
    newBasicBlock_label:
        //end, new basicBlock 分支结束,新增基本块
\end{minted}

\subsubsection{循环}
对于如下循环:
\begin{minted}{java}
    while(cond){
        //whileStmt
    }
    //end, new basicBlock
\end{minted}

按照如下方式生成

\begin{minted}{java}
    whileCond_begin:
    temp = cond;//temp保存表达式cond的计算结果
    br temp label whileBody_begin label newBasicBlock_label
    whileBody_begin:
        //whileStmt
        br label whileCond_begin
    newBasicBlock_label:
        //end, new basicBlock 循环结束,新增基本块
\end{minted}

\subsection{数组的访问}
数组的访问需要先通过一条\texttt{ElementPtr}计算出指向访问位置的地址,再通过\texttt{PointerOp}来存取数组中的值。

此处需要注意的是普通数组和函数参数中的数组访问有所区别。
函数参数中的数组的第一维度信息缺失,实质上是一个指向原数组的指针,比数组少一维。

\subsection{短路求值}

\subsubsection{分支的\&\&短路}
分支的\&\&短路可以变换为
\begin{minted}{java}
    //变换前
    if (a && b) {
        //ifStmt
    } else {
        //elseStmt
    }
    //变换后
    if (a) {
        if (b){
            //ifStmt
        } else {
            //elseStmt
        }
    } else {
        //elseStmt
    }

\end{minted}


\subsubsection{分支的||短路}

分支的||短路可以变换为
\begin{minted}{java}
    //变换前
    if (a || b) {
        //ifStmt
    } else {
        //elseStmt
    }
    //变换后
    if (a) {
        //ifStmt
    } else {
        if (b){
            //ifStmt
        } else {
            //elseStmt
        }
    }

\end{minted}


\subsubsection{循环的短路}

若循环存在短路求值,则可进行以下转化
\begin{minted}{java}
    //变换前
    while (cond) {
        //whileStmt
    }
    //变换后
    while (1) {
        if (cond) {
            //whileStmt
        } else {
            break;
        }
    }
\end{minted}


\section{编码后的修改}
由于中间代码的设计不同,代码生成的实现方式非常多样,因此此处的bug大多都是个性问题,数量多且关乎细节,此处抽象的列举几处修改。
\begin{enumerate}
    \item printf中参数的计算和字符串输出顺序问题,应先算完所有的参数再一起输出,
    不然可能会出现参数中包含函数调用,输出内容被函数内的输出阻断的效果。
    \item 短路求值转化时block种类问题。
    \item 短路求值转化时符号表问题。
    \item 短路求值转化未将后续使用的node替换为转化后的node导致条件丢失。
    \item 定义使用顺序问题,使用了本层后续定义的变量。
\end{enumerate}